% Chapter Template

\chapter{Professional, Legal, Ethical and Social issues} % Main chapter title

\label{Chapter4} % For referencing this chapter elsewhere, use \ref{Chapter4}

\lhead{Chapter 4. \emph{PLES}}

%----------------------------------------------------------------------------------------
%	SECTION 1
%----------------------------------------------------------------------------------------

\section{Professional Issues}

The experimental process to obtain the results will establish specific hardware and software configurations. The specific configurations will be run against each other in order to establish guidelines. Those guidelines will take in account the evaluated parameters and draw conclusions on them and their impact on the overall system and performance.

Each piece of the tool and its implications will be tested and evaluated against other results from the literature. The configurations, if stated specifically, will help future research avoid duplication. Potential researchers can consider parameters that were overlooked in this study. Every experiment will be ensured to be reproducible and scientifically sound. Credible work has to be proposed in order to be taken in account by the literature

%----------------------------------------------------------------------------------------
%	SECTION 2
%----------------------------------------------------------------------------------------

\section{Legal Issues}

The project will be fully developed under a GNU GPLv3 license \cite{GNUGPL}. The objective of this license is to allow complete use of the source code and modifications. It allows:

\begin{itemize}
  \item \textbf{Commercial use}: The software can be used for commercial purposes.
  \item \textbf{Distribution:} The software may be distributed.
  \item \textbf{Modification:} The software may be modified.
  \item \textbf{Patent use:} The license provides an express grant of patent rights from the contributors.
  \item \textbf{Private use:} The software may be used and modified in private.
\end{itemize}

Under the following conditions:

\begin{itemize}
  \item \textbf{Disclose source:} Source code must be made available when the software is distributed.
  \item \textbf{License and copyright notice:} A copy of the license and copyright must be included with the software.
  \item \textbf{Same license:} Modifications must be released under the same license when distributing the software.
  \item \textbf{State changes:} Changes to the code must be documented.
\end{itemize}

The limitations of this license for the original developer can be met when looking at liability or warranty. This license includes a limitation of liability and explicitly states that no warranties are provided.


%----------------------------------------------------------------------------------------
%	SECTION 3
%----------------------------------------------------------------------------------------

\section{Ethical Issues}

Providing guidelines on how reduced precision can impact performance will also highlight which parameters have an overall higher impact on the final performance. Since the goal of the study is to find the smallest size of each of the parameter without reducing the final precision too much and increasing the overall performance. Ranking parameters will give potential attackers hints on the importance of parameters. An alteration in the precision of a specific parameter can have a much higher impact when following the provided guidelines.

Moreover, since FPGAs are reconfigurable pieces of hardware, each one of the design will be tailored to the specific needs of the application. This means no superfluous space will be allocated and that overflow will have a considerable impact on the system. The use of reduced precision will have to block any attempt to downgrade precision or access restricted spaces of the hardware.

Since FPGAs are slowly developing their commercial use, fewer security measures are actually put in action. The project will have to state how precautions can be taken and how to protect the system against some attacks on precision or system integrity.

%----------------------------------------------------------------------------------------
%	SECTION 4
%----------------------------------------------------------------------------------------

\section{Social Issues}

Machine learning and especially deep learning can be applied to nearly all computationally-demanding actions. Including speech processing, image recognition or object detection for example. These applications are more and more demanded and demanding in terms of performance and portability. Any improvement to the network setup and training process will have a heavy impact on any deep learning application.

Tailoring Neural Networks on hardware of restricted size helps portability and will be proven to increase performance. This means that the implications of the project will have a heavy impact on how to design an implication. Related works providing network frameworks on FPGAs \cite{Zhao2016, Colangelo2018, Jahanshahi2019} make the most performant designs accessible to as many people as possible for their application. However, as it is the case with every novel implementation, security guidelines have to be established. Potential flaws have to be detected and included in guidelines, early in the development process. FPGAs will gain increasing attention in the following years and this means stricter protections should be presented.
