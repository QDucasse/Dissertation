\chapter{Introduction} % Main chapter title

\label{Chapter1} % For referencing this chapter elsewhere, use \ref{Chapter1}

\lhead{Chapter 1. \emph{Introduction}}

%----------------------------------------------------------------------------------------
%	SECTION 1 - Context
%----------------------------------------------------------------------------------------

\section{Context}

Ever since the creation of computers, their computation power has been a source of progress in terms of high-precision scientific computations in one hand or efficient energy expensive mass computations. In one case or the other, this performance has been met through the development of more and more performant hardware as stated by Moore through his law announced in 1965. However, the last decade has proven to counterbalance this law and make it meet an end due to physical and thermal limitations. This decrease in hardware evolution has however shifted the focus to other vectors of performance. One of them is parallelisation through the use of multiple or a cluster of computers to increase the computational power. This focus has enabled the development of new architectures tailored to those specific tasks, such as Graphical Processing Units (GPUs) or reprogrammable architectures such as Field Programmable Gate Arrays (FPGAs). Another field of interest, that will be the main focus of this report is the use of reduced precision to increase the performance of computational-expensive tasks. This report will look over the ideas under mixed-precision, the ways to implement them efficiently, applications that can derive from those methods and how to translate them in physical architectures.

%-----------------------------------
%	SECTION 2 - Aims and Objectives
%-----------------------------------
\section{Aims and Objectives}

The aim of the report is to give to the reader an insight of the mixed-precision mindset and intentions by covering important articles and ideas the literature holds. The practical aim after looking into the literature is to implement an application of mixed-precision to demonstrate how it can provide hardware acceleration. The choice of the application, the architecture and the methods it will be done has to be carried out and explain through the report.

A Proof of Concept of a mixed-precision will be developed later on during the \emph{Dissertation Thesis}, which parts of the given report will be part of. A better understanding of the literature surrounding the concept of mixed-precision and its applications is expected and should result in the development and implementation of a Convolutional Neural Network on an FPGA.

The resulting aims are the following. The objectives will be defined after the literature review in the "Requirements analysis" chapter.

\textbf{Aims}
\begin{enumerate}
  \item Determine the motivations behind mixed-precision.
  \item Assess the impact of mixed-precision on state-to-the-art applications.
  \item Present the methods to apply a mixed-precision mindset to a given application.
  \item Look into the range of applications mixed-precision has in the literature.
  \item Provide insight on the choice of architecture for mixed-precision applications.
\end{enumerate}

%-----------------------------------
%	SECTION 2 - Structure of the report
%-----------------------------------

\section{Structure of the report}

The given report is structured as follows. Section 1 provides a brief \emph{Introduction} to the context and objectives of the report. Section 2 consists of the \emph{Literature Review}, presenting the background, ideas, methods and applications of mixed-precision. Section 3 consists of a \emph{Requirement Analysis} of the latter project of the dissertation. Section 4 presents any \emph{Professional, Legal, Ethical and Social Issues} the project can highlight. Section 5 will display the proposed \emph{Project Plan} the dissertation project will be conducted along. Finally, Section 6 consists of the \emph{Conclusion} of the report.
