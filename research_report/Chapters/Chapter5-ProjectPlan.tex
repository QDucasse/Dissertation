\chapter{Project Plan}

\label{Chapter5} For referencing this chapter elsewhere, use \ref{Chapter5}

\lhead{Chapter 5. \emph{Project Plan}}

%----------------------------------------------------------------------------------------
%	SECTION 1
%----------------------------------------------------------------------------------------

\section{Project Schedule}

The schedule of the project will consist of a list of tasks along with their estimated time. The best representation in order to output a graphical representation of those tasks and their deadline is to use a \emph{Gantt Chart}. This chart enables a graphical representation of the plan along with the allocated resources for each one of the tasks. This will allow to identify the critical path of the project - the sequence of tasks from beginning to end that takes the longest time to complete, any delay on one of the tasks of the sequence will automatically dealy the whole project.

The tasks that will need to be completed by the end of the dissertation are the following:

\begin{itemize}
  \item Appropriation of the FPGA hardware architecture
    \Subitem Basic actions
    \Subitem Complete cycle
    \Subitem Higher-level implementation
  \item Appropriation of the CNN
    \Subitem Basic implementation
    \Subitem Parameter tuning
    \Subitem Optimisation implementation
  \item Combination of the CNN and FPGA
  \item Backend software
    \Subitem Coordination between the two elements (hardware and network)
    \Subitem GUI
    \Subitem Security rules
  \item Experimentation guidelines
  \item Experimentation results
  \item Dissertation writing
\end{itemize}

%----------------------------------------------------------------------------------------
%	SECTION 2
%----------------------------------------------------------------------------------------

\section{Supporting Plans and Risk Management}

\subsection{Supporting Plans}

Internship.

\subsection{Risk Management}

What risks are there?

How likely are there to occur?

What will their impact be?

How can we minimise their occurrence?
